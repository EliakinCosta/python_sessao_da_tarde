\begin{frame}{Funções}

\begin{itemize}
	\item São definidas pela palavra reservada \textit{def}
	\item Podem possuir parâmetros default
	\item São objetos de primeira classe
	\begin{itemize}
		\item Criada em tempo de execução
		\item Pode ser atribuída a uma variável
		\item Passada como argumento para uma função		
		\item Devolvida como resultado de uma função
	\end{itemize}
\end{itemize}
\end{frame}


\begin{frame}{Funções}

\centering \text{Exemplo de função}
\lstset{language=Python}
\lstinputlisting{code/exemplo_funcao.py}
\end{frame}

\begin{frame}{Classes}

\begin{itemize}	
	\item São definidas com a palavra reservada \textit{class}
	\item Python suporta herança múltipla
	\item Python possui sobrecarga de operadores, assim como C++
	\item Todos os atributos são públicos em Python
	\item Python não possui uma palavra reservada para interface como JAVA, mas o python tem classes abstratas.
	
\end{itemize}

\end{frame}

\begin{frame}{Classes }
\centering \text{Exemplo de classe}
\lstset{language=Python}
\lstinputlisting[
basicstyle=\tiny, %or \small or \footnotesize etc.
]{code/exemplo_classe.py}
\end{frame}
\begin{frame}{Estruturas básicas}
    \begin{alertblock}{Atenção!}
        Antes de começar, uma informação importante!\\
        O \LaTeX é \textit{case sensitive}!!!
    \end{alertblock}
\end{frame}  

\begin{frame}{Estruturas básicas}
    \begin{itemize}
        \item A base dos comandos \LaTeX são iniciados com "$\backslash$"  (barra invertida)
        \item Exemplo $\backslash$comando[arg opt]$\{$ $\}$
        \item \textbf{Preâmbulo} é a parte inicial do documento fonte, é a área de que define os estilos e características do documento
        \item \textbf{Corpo do Documento}, como o nome sugere, é a parte onde está toda parte relacionada ao documento em si (seções, parágrafos, sumário, recursos gráficos...)
        \item \textbf{Referências}, apesar de ser definida (geralmente) no final do corpo do documento, a referência é uma recurso a parte da ferramenta e iremos ver detalhadamente mais a frente
    \end{itemize}
\end{frame}

\begin{frame}{Estruturas básicas: Corpo documento}
    \begin{block}{}
        $\backslash$documentclass[opcoes]$\{$estilo do documento$\}$
        $\backslash$begin$\{$document$\}$\\
        $...$\\
        $\backslash$end$\{$document$\}$
    \end{block}
\end{frame}

\begin{frame}{Estruturas básicas: Opções do documento}
    \begin{block}{}
        $\backslash$documentclass[papel, fonte, colunas]$\{$estilo do documento$\}$
    \end{block}
    \begin{itemize}
        \item tipo do papel: a4paper, letterpaper, a5paper, b5paper
        \item tamanho da fonte: 10pt, 11pt, 12pt, ...
        \item colunas: onecolumn, twocolumn
    \end{itemize}
\end{frame}


\begin{frame}{Estruturas básicas: Outras opções}
    \begin{block}{}
        $\backslash$documentclass[papel, fonte, colunas]$\{$estilo do documento$\}$
    \end{block}
    \begin{itemize}
        \item \textbf{landscape}: Orientação da Página;
        \item \textbf{titlepage}: Página de título;
        \item \textbf{leqno}: Numeração das fórmulas à esquerda (ao invés do padrão à direita);
        \item \textbf{fleqn}: Alinhamento das fórmulas à esquerda (ao invés do padrão centralizado);
        \item \textbf{openright}: Capítulos são iniciados apenas nas páginas ímpares;
        \item \textbf{openany}: Capítulos são iniciados em páginas pares ou ímpares
        
    \end{itemize}
\end{frame}


\begin{frame}{Estruturas básicas: Estilo do documento}
    \begin{itemize}
        \item article
        \item report
        \item book
        \item letter
        \item beamer \footnote{apresentação}
        \item paper
        \item amsart
        \item amsbook
        \item amsproc
        \item proc
        \item coursepaper 
        \item entre muitos outros
    \end{itemize}
\end{frame}


\begin{frame}{Estruturas básicas: Exemplo}
    \begin{block}{}
        $\backslash$documentclass[a4paper, 12pt, twocolumn]$\{$article$\}$\\
            $\backslash$begin$\{$document$\}$\\
                Texto do documento\\
            $\backslash$end$\{$document$\}$
    \end{block}
\end{frame}

\begin{frame}{Estruturas básicas: Pacotes}
    \begin{itemize}
        \item Um dos melhores recursos do \LaTeX, porém pode se tornar problemático...
        \item Conjunto de arquivos que implementam características adicionais para os documentos escritos em \LaTeX
        \item  Para documentos mais elaborados --> comandos básicos não são suficientes
        \item Alguns pacotes já vêm como distribuição básica do \LaTeX
        \item Os demais podem ser encontrados separadamente
    \end{itemize}
\end{frame}

\begin{frame}{Estruturas básicas: Pacotes}
    \begin{itemize}
        \item Os pacotes são inseridos no \underline{Preâmbulo} do documento, utilizando o seguinte comando:
    \end{itemize}
    \begin{block}{}
        $\backslash$usepackage[opções]$\{$nomePacote$\}$\\
    \end{block}
\end{frame}

\begin{frame}{Estruturas básicas: Principais pacotes \footnote{Importante: É necessário verificar a compatibilidade dos pacotes com a versão\\ do \LaTeX}}
    \begin{itemize}
        \item \textbf{color}: Para usar cores no texto;
        \item \textbf{babel}: Para traduzir termos que aparecem em inglês na estrutura do documento. Use a opção \textbf{[brazil]}.
        \item \textbf{fontenc}: Permite que o \LaTeX a acentuação feita direto pelo teclado. É usado com o opcional [T1].
        \item \textbf{amsfonts}: Define alguns estilos de letras para o ambiente matemático;
        \item \textbf{graphicx}: Para usar gráficos no documento.
        
    \end{itemize}
\end{frame}


\begin{frame}{Estruturas básicas: Espaçamento}
    \begin{itemize}
        \item Não está relacionado ao número de vezes que apertamos o "espaço"
        \item No \LaTeX sempre será considerado \textbf{um espaço}!
        \item Caso precise \ \ \  mais de um (acho que não): $\backslash $ $\backslash $ $\backslash $
    \end{itemize}
\end{frame}


\begin{frame}{Estruturas básicas: Parágrafo e Quebra de Linha}
    \begin{itemize}
        \item Para criar um novo parágrafo, basta pular uma linha $<$ENTER$>$ ou utilizar o  comando $\backslash par$
        \item O número de linhas "saltadas" não está relacionado ao número de vezes em que apertamos a tecla $<$ENTER$>$
        \item No \LaTeX, isso não importa --> sempre será considerado apenas um!
        \item O espaçamento é controlado pelo estilo do documento
        \item Para inserir uma quebra de linha use $\backslash \backslash$ ou ainda o comando $\backslash newline$.
    \end{itemize}
\end{frame}

\begin{frame}{Estruturas básicas: Parágrafo e Quebra de Linha}
    \begin{itemize}
        \item  Mais alguns comandos e coisas Interessantes
        \begin{itemize}        
            \item $\backslash$linebreak[$n$]: Força a quebra de linha
            \item $\backslash$nolinebreak[$n$]: Ajusta o texto de forma a ignorar uma            possível quebra de linha
            \item $\backslash$pagebreak[$n$]: Força a quebra de página
            \item $\backslash$nopagebreak[$n$]: Ajusta o texto de forma a ignorar uma
            possível quebra de página
        \end{itemize}
    \end{itemize}
    \begin{block}{Importante!}
        O argumento [$n$] pode ser um valor entre 0 e 4.\\ Se $n$ $<$ 4 o \LaTeX pode ignorar o comando se o resultado for muito ruim.
    \end{block}
\end{frame}

\begin{frame}{Estruturas básicas: Parágrafo e Quebra de Linha}
    \begin{itemize}
        \item  As consequências da quebra de linha:
        \begin{itemize}        
            \item \textbf{overfull box}: Quando o \LaTeX não encontra nenhuma possibilidade satisfatória para produzir parágrafos totalmente retos (alinhamento justificado), então uma das linhas fica maior que as demais (muito comprida)
            \item Isto acontece quando o \LaTeX não consegue adicionar um hífen (separação silábica)
            \item Uma alternativa é usar o comando $\backslash$sloppy para aumentar o espaçamento entre as palavras
            \item O resultado final não é o melhor, mas é totalmente aceitável na maioria das vezes            
        \end{itemize}
    \end{itemize}
\end{frame}

\begin{frame}{Estruturas básicas: Elementos de texto}
    \begin{alertblock}{Vamos parar essa etapa por aqui... }
        Essa etapa iremos continuar amanhã... \\
        Algumas \textit{nuances} deste tema precisam ser vista com calma... =$)$
    \end{alertblock}
\end{frame}
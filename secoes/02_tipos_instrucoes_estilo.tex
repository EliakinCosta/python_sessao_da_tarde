\begin{frame}{Tipos Básicos do python}
\begin{itemize}
	\item Númericos:
	\begin{itemize}
		\item inteiro (int)
		\item ponto flutuante (float)
		\item booleano (bool) 1 ou 0
		\item complexo (complex)
	\end{itemize}
\end{itemize}
\begin{itemize}
		\item Obs:
		\begin{itemize}
			\item Suportam as operações básicas, mod, etc (+, -, *, /, \%, **, +=, -=, *=, /=,\%=, **=).
			\item Operações entre diferentes tipo irão resultar no tipo mais complexo
			\item Cuidado, números inteiros tem precisão infinita
		\end{itemize}
\end{itemize}
\end{frame}

\begin{frame}{Tipos Básicos do python}
\begin{itemize}
	\item Iteráveis:
	\begin{table}[]
		\centering
		\caption{Iteráveis}
		\label{my-label}
		\begin{tabular}{|l|c|c|l|l|}
			\hline
			\multicolumn{1}{|c|}{} & Ordenada              & Modificável           & \multicolumn{1}{c|}{Unicidade}  \\ \hline
			strings                   & \multicolumn{0}{c|}{X} &                           &                       \\ \hline
			listas                 & X                     & X                     &                                          \\ \hline
			tuplas                 & X                     & \multicolumn{1}{l|}{} &                    \\ \hline
			sets                   & \multicolumn{1}{l|}{} & X                     & \multicolumn{1}{c|}{X}                             \\ \hline
		\end{tabular}
	\end{table}
	\item Vamos tentar um pouco de código
\end{itemize}
		\textbf{\text{Obs: Iremos focar apenas em listas}}
\end{frame}
% 2015-05-21 - Emerson Ribeiro de Mello - mello@ifsc.edu.br
% \documentclass[handout,xcolor=pdftex,dvipsnames,table]{beamer}
\documentclass{beamer}

\usepackage[utf8]{inputenc}
\usepackage[T1]{fontenc}
\usepackage[portuguese]{babel}

% usando tema personalizado. 
% arquivo beamerthemeIFSC.sty deve estar no mesmo diretório do .tex
\usepackage{beamerthemeIFSC}
\usepackage{tikz}
\usepackage{xargs}
\usepackage{amsmath}
\usepackage{mathtools}
\usepackage{soul} %fazer texto tachado
\usepackage{epigraph}
\usepackage[colorinlistoftodos,prependcaption,textsize=tiny]{todonotes}
\usepackage{graphicx}
\usepackage{listings}
\usepackage{color}

\definecolor{dkgreen}{rgb}{0,0.6,0}
\definecolor{gray}{rgb}{0.5,0.5,0.5}
\definecolor{mauve}{rgb}{0.58,0,0.82}

\lstset{frame=tb,
	language=Java,
	aboveskip=3mm,
	belowskip=3mm,
	showstringspaces=false,
	columns=flexible,
	basicstyle={\small\ttfamily},
	numbers=none,
	numberstyle=\tiny\color{brown},
	keywordstyle=\color{blue},
	commentstyle=\color{dkgreen},
	stringstyle=\color{mauve},
	breaklines=true,
	breakatwhitespace=true,
	tabsize=3
}

\usepackage{xargs}
%
\newcommandx{\unsure}[2][1=]{\todo[linecolor=red,backgroundcolor=red!25,bordercolor=red,#1]{#2}}
\newcommandx{\info}[2][1=]{\todo[linecolor=OliveGreen,backgroundcolor=OliveGreen!25,bordercolor=OliveGreen,#1]{#2}}
%

\hypersetup{pdfstartview={Fit},pdftitle={\@title},
 	pdfsubject={IFBA},pdfauthor={\@author}
}

\hyphenation{ob-je-ti-vo}


%%%%%%%%%%%%%%%%%%%%%%%%%%%%%%%%%%%%%%%%%%%%


\title{Python}
\subtitle{Programando de forma idiomática}
\author{Eliakin Costa}
\date{26 de abril de 2017}
\institute{Instituto Federal da Bahia\\
campus Salvador\\
\url{eliakincosta@ifba.edu.br}
}



%%%%%%%%%%%%%%%%%%%%%%%%%%%%%%%%%%%%%%%%%%%%

\begin{document}

\begin{frame}[t]
	\maketitle
\end{frame}

% Descomente as linhas abaixo se desejar colocar um sumário de todas as seções
\begin{frame}[t]{Agenda}
\tableofcontents
\end{frame}

% O trecho de código abaixo serve para sobrescrever regras do template
\def\sectionname{}
\def\insertsectionnumber{}
\def\subsectionname{}
\def\insertsubsectionnumber{}

\AtBeginSection{\frame{\sectionpage}\addtocounter{framenumber}{-1}}


\AtBeginSubsection{\frame{\subsectionpage}\addtocounter{framenumber}{-1} }
\AtBeginSubsubsection{\frame{\subsubsectionpage}\addtocounter{framenumber}{-1} }



%%%%%%%%%%%%%%%%%%%%%%%%%%%%%%%%%%%%%%%%%%%%
% Inicio do documento
%%%%%%%%%%%%%%%%%%%%%%%%%%%%%%%%%%%%%%%%%%%%

\begin{frame}{Quem sou eu?}
    \begin{itemize}
        \item Meu nome é Eliakin Costa,
        \item Sou aluno de ADS (IFBA),
        \item Apaixonado pelo software livre $\backslash$o/
        \item Membro e Colaborador do OPAI e KDE
        \item \textbf{Contatos}:        
        \begin{itemize}
            \item \textbf{E-mail}: eliakim170 at gmail.com
            \item \textbf{GitHub}: github.com/eliakincosta
        \end{itemize}
    \end{itemize}
\end{frame}
%

\section{Introdução}
\begin{frame}{O começo de tudo}
    \begin{itemize}
        \item  CWI - Centro Tecnológico
        \item Guido Van Rossum
        \item Linguagem ABC        
    \end{itemize}
\end{frame}
%
\begin{frame}{Python!? Que nome é esse? }
	\begin{center}
		\includegraphics[scale=0.5]{img/python.png}
	\end{center}
\end{frame}

\begin{frame}{Python!? Que nome é esse? }
    \begin{itemize}
        \item Python Flying in Circus
        \item Episódio do spam xD
		\item Guido não gostava da relação do python com uma cobra        
    \end{itemize}
\end{frame}
%

\begin{frame}{Python!? Que nome é esse? }
\begin{center}
	\includegraphics[scale=1.0]{img/python_book.jpg}
\end{center}
\end{frame}

\begin{frame}{Guido Van Rossum - O ditador benevolente}
\begin{itemize}
	\item Está vivo!!!!
	\item Google de 2005-2013
	\item Dropbox de 2013-hoje
\end{itemize}
\end{frame}

\begin{frame}{Por que eu devo aprender Python?}
\begin{itemize}
        \item Linguagem de Propósito geral
        \item Simples e intuitivo
        \item Comunidade espetacular
        \item Você consegue fazer quase tudo apenas com os recursos da  própria linguagem
        \item Multiparadigma
    \end{itemize}
\end{frame}
%

\begin{frame}{Quem está usando python?}
\begin{figure}[htp]
	\centering
	\includegraphics[width=.3\textwidth]{img/bit_torrent.jpg}\quad
	\includegraphics[width=.3\textwidth]{img/globo.png}\quad
	
	\medskip
	\medskip
	\medskip
	\includegraphics[width=.3\textwidth]{img/google.png}\quad
	\includegraphics[width=.3\textwidth]{img/youtube.png}
	
	\medskip
\end{figure}

\end{frame}
%

\begin{frame}{Quem está usando python?}
\begin{figure}[htp]
	\centering
	\includegraphics[width=.3\textwidth]{img/blender.jpg}\quad
	\includegraphics[width=.3\textwidth]{img/gimp.png}\quad
	\includegraphics[width=.3\textwidth]{img/krita.png}\quad
\end{figure}

\end{frame}


\section{Tipagem Dinâmica, mas nem tanto}
\begin{frame}{Características da Tipagem Dinâmica}

\begin{itemize}
	\item Verificações de tipo feitas em runtime
	\item Verificações feita sobre o valor(ex: class int)
	\item Bob Harper, um dos criadores da linguagem Standard ML, "uma linguagem dinamicamente tipada é uma linguagem estaticamente tipada com apenas um tipo estático".
\end{itemize}
\end{frame}

\begin{frame}{Como é a tipagem estática?}

\centering \text{JAVA}
\lstset{language=Java}
\lstinputlisting{code/tipagem_estatica.java}
      
\end{frame}

\begin{frame}{Como é a tipagem dinâmica?}

\centering \text{Python}
\lstset{language=Python}
\lstinputlisting{code/tipagem_dinamica.py}

\end{frame}

\begin{frame}{Tipagem dinâmica, mas forte}
\begin{itemize}
	\item Observe o seguinte código em php:
	\lstset{language=php}
	\lstinputlisting{code/tipagem_fraca.php}
\end{itemize}
\begin{itemize}
	\item Vamos testar o mesmo código em python.
\end{itemize}

\end{frame}

\begin{frame}{Tipagem forte}
	\begin{itemize}
		\item Código mais robusto
		\item Tipos inválidos ocasionarão exceções
		\item Explícito é melhor do que implícito
	\end{itemize}
\end{frame}


\section{Tipos, Instruções e Estilo}
\begin{frame}{Tipos Básicos do python}
\begin{itemize}
	\item Númericos:
	\begin{itemize}
		\item inteiro (int)
		\item ponto flutuante (float)
		\item booleano (bool) 1 ou 0
		\item complexo (complex)
	\end{itemize}
\end{itemize}
\begin{itemize}
		\item Obs:
		\begin{itemize}
			\item Suportam as operações básicas, mod, etc (+, -, *, /, \%, **, +=, -=, *=, /=,\%=, **=).
			\item Operações entre diferentes tipo irão resultar no tipo mais complexo
			\item Cuidado, números inteiros tem precisão infinita
		\end{itemize}
\end{itemize}
\end{frame}

\begin{frame}{Tipos Básicos do python}
\begin{itemize}
	\item Iteráveis:
	\begin{table}[]
		\centering
		\caption{Iteráveis}
		\label{my-label}
		\begin{tabular}{|l|c|c|l|l|}
			\hline
			\multicolumn{1}{|c|}{} & Ordenada              & Modificável           & \multicolumn{1}{c|}{Unicidade}  \\ \hline
			strings                   & \multicolumn{0}{c|}{X} &                           &                       \\ \hline
			listas                 & X                     & X                     &                                          \\ \hline
			tuplas                 & X                     & \multicolumn{1}{l|}{} &                    \\ \hline
			sets                   & \multicolumn{1}{l|}{} & X                     & \multicolumn{1}{c|}{X}                             \\ \hline
		\end{tabular}
	\end{table}
	\item Vamos tentar um pouco de código
\end{itemize}
		\textbf{\text{Obs: Iremos focar apenas em listas}}
\end{frame}

\section{Funções e Classes}
\begin{frame}{Funções}

\begin{itemize}
	\item São definidas pela palavra reservada \textit{def}
	\item Podem possuir parâmetros default
	\item São objetos de primeira classe
	\begin{itemize}
		\item Criada em tempo de execução
		\item Pode ser atribuída a uma variável
		\item Passada como argumento para uma função		
		\item Devolvida como resultado de uma função
	\end{itemize}
\end{itemize}
\end{frame}


\begin{frame}{Funções}

\centering \text{Exemplo de função}
\lstset{language=Python}
\lstinputlisting{code/exemplo_funcao.py}
\end{frame}


\begin{frame}{Decoradores }
\begin{center}
	\centering \text{Exemplo de decorador}
	\includegraphics[scale=0.4]{img/decorator_example.png}
\end{center}
\end{frame}

\begin{frame}{Decoradores}

\begin{itemize}
	\item São \textit{callables} que aceitam função como parâmetro 
	\item Criada em tempo de importação
	\item Mudar comportamento de funções ou métodos		
	\item Exemplo real de autenticação no Django
	\item Explícito melhor do que implícito
\end{itemize}

\end{frame}

\begin{frame}{Classes - Classe Carro}

\begin{itemize}
	\item São abstrações de coisas ou comportamentos do mundo real
	\item São definidas com a palavra reservada \textit{class}
	\item Python suporta herança múltipla
	\item Python possui sobrecarga de operadores, assim como C++
	\item Todos os atributos são públicos em Python
	\item Python não possui uma palavra reservada para interface como JAVA, mas o python tem classes abstratas.
	
\end{itemize}

\end{frame}

\begin{frame}{Classes }
\begin{center}
	\includegraphics[scale=0.5]{img/class_example.png}
\end{center}
\end{frame}

\section{Ainda Tem muito mais}
\begin{frame}{Ainda tem muito mais...}
	\begin{itemize}
		\item Descriptors
		\item Classes abstratas
		\item Metaclasses
		\item Toda a biblioteca padrão
	\end{itemize}
\end{frame}

%Finalizando
\begin{frame}{Peguntas}
	\includegraphics[scale=0.5]{img/perguntas.jpg}
\end{frame} 
 
\begin{frame}{Referências}
 	\begin{itemize}
 		\item Python Fluente - Luciano Ramalho
 		\item Learning Python - Mark Lutz
 		\item Python Tutorial https://docs.python.org/3/tutorial/
 	\end{itemize}
\end{frame} 


\end{document}